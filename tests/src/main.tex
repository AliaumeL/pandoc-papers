\section{Main}

\subsection{Testing Environments}

The environments allowed should at least contain

\begin{enumerate}
  \item Theorem
  \item Definition
  \item Lemma
  \item Proposition
  \item Corollary
  \item Example
\end{enumerate}

\foreach \x in {theorem,definition,lemma,proposition,corollary,example} {
  \begin{\x}[name={Sample \x}]
    This is a sample \x\ to demonstrate \x\ environments in LaTeX.
  \end{\x}
}

\subsection{Testing proofs}

Here is a sample theorem with a proof.
\begin{theorem}[name={Sample Theorem with Proof}]
  This is a sample theorem that includes a proof.
\end{theorem}
\begin{proof}
  This is the proof of the sample theorem. It demonstrates how to include a proof environment in LaTeX.
  \begin{enumerate}
    \item test
    \item test
  \end{enumerate}
\end{proof}

We can use \texttt{\textbackslash qedhere} to place the QED symbol at a specific location within the proof.
\begin{theorem}[name={Sample Theorem with QED Here}]
  This is another sample theorem that includes a proof with QED here.
\end{theorem}
\begin{proof}
  \begin{itemize}
    \item First step of the proof.
    \item Second step of the proof. \qedhere
  \end{itemize}
\end{proof}


\subsection{Testing Knowledge Macros}

\AP 
A \intro{monoid} is a
set $M$ equipped with a binary operation $\cdot : M \times M \to M$
and an identity element $e \in M$ such that for all $a, b, c \in M$,
\begin{itemize}
  \item (Associativity) $(a \cdot b) \cdot c = a \cdot (b \cdot c)$
  \item (Identity) $e \cdot a = a \cdot e = a$
\end{itemize}
A \intro(monoid){morphism} between two monoids $(M, \cdot_M, e_M)$ and $(N, \cdot_N, e_N)$
is a function $f : M \to N$ such that for all $a, b \in M$,
\begin{itemize}
  \item $f(a \cdot_M b) = f(a) \cdot_N f(b)$
  \item $f(e_M) = e_N$
\end{itemize}

\AP
A \intro{group} is a monoid $(G, \cdot, e)$ such that for every element $a \in G$,
there exists an inverse element $a^{-1} \in G$ satisfying
\begin{itemize}
  \item $a \cdot a^{-1} = a^{-1} \cdot a = e$
\end{itemize}
A \intro(group){morphism} between two groups $(G, \cdot_G, e_G)$ and $(H, \cdot_H, e_H)$
is a function $g : G \to H$ such that for all $a, b \in G$,
\begin{itemize}
  \item $g(a \cdot_G b) = g(a) \cdot_H g(b)$
  \item $g(e_G) = e_H$
\end{itemize}


We can distinguish between a \kl(monoid){morphism} and a \kl(group){morphism}.

\AP We use $\intro*\omegaOrd$ to denote the first infinite ordinal. And can
refer to it using $\omegaOrd$.


\subsection{Testing Cross-References}

As shown in \cref{thm:sample}, this is a sample theorem to demonstrate theorem environments in LaTeX.

We can also cite papers such as \cite{paper1,paper2}.
